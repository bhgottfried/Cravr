\documentclass[11pt]{article}


%%% Packages
%%
\usepackage{amsmath}
\usepackage{amsfonts}
\usepackage{amssymb}
\usepackage{fancyhdr}
\usepackage{float}
\usepackage{graphicx}
\usepackage{listings}
\usepackage{enumitem}
\usepackage{url}
\usepackage[margin = 1in, headheight = 13.6pt]{geometry}
\usepackage[linktoc=all]{hyperref}
%%
%%%


%%% Formatting
%%
\parindent 0em
\parskip 1ex
\pagestyle{fancy}
\fancyhead{}
\fancyfoot{}
\fancyhead[L]{\slshape\MakeUppercase{{\myTitle}}}
\fancyhead[R]{\slshape{\myName}}
\fancyfoot[C]{\thepage}
\setcounter{tocdepth}{3}
%%
%%%


%%% User defined variables
%%
\def \myTitle {Cravr API Docs}
\def \myName {Ben Gottfried, Elias Talcott, and Justin Hyndman}
\def \myDate {Last Revised: April 7, 2021}
%%
%%%

%%% Custom commands
%%
\newcommand*{\tabulardef}[3]{\begin{tabular}[t]{@{}lp{\dimexpr\linewidth-#1}@{}}
    #2&#3
  \end{tabular}}
\newlist{deflist}{description}{2}
\setlist[deflist]{labelwidth=2cm,leftmargin=!,font=\normalfont}
%%
%%%


\begin{document}

\begin{titlepage}
\title{\myTitle}
\author{\myName}
\date{\myDate}
\maketitle
\tableofcontents
\thispagestyle{empty}
\end{titlepage}


\section{Backend Function Index}
\subsection{Authentication}
Logs in existing users and creates new users. Located in \url{authentication\_utils.py}.

\subsubsection{authenticate\_user}
\textbf{Description}

Log in an existing user. Fails if username/password do not match a database entry or if database connection cannot be established.

\textbf{Arguments}
\begin{deflist}
	\item[username]String containing username. Must be a valid email address.
	\item[password]String containing password.
\end{deflist}

\textbf{Return Value}

True if authentication is successful. False otherwise.

\subsubsection{register\_user}
\textbf{Description}

Create a new user in the database and log in. Fails if username is already associated with an account or if database connection cannot be established.

\textbf{Arguments}
\begin{deflist}
	\item[username]String containing username. Must be a valid email address.
	\item[password]String containing password.
\end{deflist}

\textbf{Return Value}

True if registration is successful. False otherwise.


\subsection{Database}
Executes reads and writes to the database. Located in \url{database\_utils.py}.

\subsubsection{setup}
\textbf{Description}

Class method to create a MySQL instance at the beginning of app execution. Uses a default configuration from a DatabaseConfig object and allows custom configuration parameters. Once configured, create one connection and add it to the connection pool.

\textbf{Arguments}
\begin{deflist}
	\item[app]Flask app instance.
	\item[**kwargs]Custom configuration parameters.
	\begin{deflist}
		\item[socket]Tuple containing hostname and port of the MySQL database. Defaults are 
					 "MYSQL\_DATABASE\_HOST" environment variable and port 3306.
		\item[credentials]Tuple containing user and password to access database.
		\item[database]Name of database to use. Default is "cravr".
		\item[charset]Character set used by database. Default is "utf8".
	\end{deflist}
\end{deflist}

\textbf{Return Value}

None.

\subsubsection{execute\_query}
\textbf{Description}

Gets a connection from the connection pool, executes a query on the database, and gives the connection back. Fails if database connection cannot be established.

\textbf{Arguments}
\begin{deflist}
	\item[query]String containing MySQL query to be executed.
\end{deflist}

\textbf{Return value}

String containing the first result of the query. -1 if database connection failed.

\subsection{Recommender}
Suggests restaurants based on user and restaurant data. Located in \url{recommender.py}

\subsubsection{get\_restaurant}
\textbf{Description}

\textbf{Arguments}
\begin{deflist}
	\item[]
\end{deflist}

\textbf{Return value}

\subsubsection{cache\_restaurant}
\textbf{Description}

\textbf{Arguments}
\begin{deflist}
	\item[]
\end{deflist}

\textbf{Return value}

\subsection{Routing}
Interacts directly with the frontend for authentication, restaurant recommendations, and user feedback. Provides restaurant/review data and routing information. Located in \url{app.py}.

\subsubsection{login}
\textbf{Description}

Uses authentication and database functionality to check if user's credentials are valid. Fails if username not found in database, hashed password does not match, or database connection cannot be established.

\textbf{Return value}

\{"result": "/"\} if authentication successful. \{"result": "/Login"\} otherwise.

\subsubsection{register}
\textbf{Description}

Uses authentication and database functionality to register a new user. Fails if username already exists in database or database connection cannot be established.

\textbf{Return value}

\{"result": "/Login"\} if registration successful. \{"result": "/Register"\} otherwise.

\subsubsection{get\_restaurant}
\textbf{Description}

Parses a user's search parameters and gives a restaurant recommendation. 

\textbf{Return value}

\{"result": RESTAURANT\_OBJECT\} where RESTAURANT\_OBJECT is the return value of the Recommender class's get\_restaurant method.

\subsubsection{rate\_suggestion}
\textbf{Description}

Take a restaurant ID along with a user's rating of "Yummy", "Maybe later", or "Yuck" to update their review list and train the recommender.

\textbf{Return value}

None.

\subsubsection{get\_reviews}
\textbf{Description}

Fetch a list of restaurants that the user needs to review.

\textbf{Return value}

\{"result": REVIEW\_LIST\} where REVIEW\_LIST is the return value of the User class's get\_reviews method.

\subsubsection{submit\_review}
\textbf{Description}

Remove a restaurant from a user's review list and use review data to train their recommender.

\textbf{Return value}

None.

\subsection{User Utilities}
Manage users' restaurant caches, review lists, and recommender training data. Located in \url{user.py} and \url{user\_data\_utils.py}



\subsection{Yelp Fusion API}
Gets restaurant data from Yelp Fusion API. Located in \url{yelp\_api\_utils.py}

\subsubsection{business\_search}
\textbf{Description}

\textbf{Arguments}
\begin{deflist}
	\item[]
\end{deflist}

\textbf{Return value}

\subsubsection{business\_details}
\textbf{Description}

\textbf{Arguments}
\begin{deflist}
	\item[]
\end{deflist}

\textbf{Return value}

\subsubsection{business\_reviews}
\textbf{Description}

\textbf{Arguments}
\begin{deflist}
	\item[]
\end{deflist}

\textbf{Return value}

\pagebreak

\section{Code Samples}
\subsection{}


\end{document}