\documentclass[11pt]{article}


%%% Packages
%%
\usepackage{amsmath}
\usepackage{amsfonts}
\usepackage{amssymb}
\usepackage{fancyhdr}
\usepackage{float}
\usepackage{graphicx}
\usepackage{listings}
\usepackage{enumitem}
\usepackage{url}
\usepackage[margin = 1in, headheight = 13.6pt]{geometry}
\usepackage[linktoc=all]{hyperref}
%%
%%%


%%% Formatting
%%
\parindent 0em
\parskip 1ex
\pagestyle{fancy}
\fancyhead{}
\fancyfoot{}
\fancyhead[L]{\slshape\MakeUppercase{{\myTitle}}}
\fancyhead[R]{\slshape{\myName}}
\fancyfoot[C]{\thepage}
\setcounter{tocdepth}{3}
%%
%%%


%%% User defined variables
%%
\def \myTitle {Cravr API Docs}
\def \myName {Ben Gottfried, Elias Talcott, and Justin Hyndman}
\def \myDate {Last Revised: April 7, 2021}
%%
%%%

%%% Custom commands
%%
\newcommand*{\tabulardef}[3]{\begin{tabular}[t]{@{}lp{\dimexpr\linewidth-#1}@{}}
    #2&#3
  \end{tabular}}
\newlist{deflist}{description}{2}
\setlist[deflist]{labelwidth=2cm,leftmargin=!,font=\normalfont}
%%
%%%


\begin{document}

\begin{titlepage}
\title{\myTitle}
\author{\myName}
\date{\myDate}
\maketitle
\vspace{-4ex}
\tableofcontents
\thispagestyle{empty}
\end{titlepage}


\section{Backend Function Index}
\subsection{Authentication}
Logs in existing users and creates new users. Located in \url{authentication\_utils.py}.

\subsubsection{authenticate\_user}
\textbf{Description}

Log in an existing user. Fails if username/password do not match a database entry or if database connection cannot be established.

\textbf{Arguments}
\begin{deflist}
	\item[username]String containing username. Must be a valid email address.
	\item[password]String containing password.
\end{deflist}

\textbf{Return Value}

True if authentication is successful. False otherwise.

\subsubsection{register\_user}
\textbf{Description}

Create a new user in the database and log in. Fails if username is already associated with an account or if database connection cannot be established.

\textbf{Arguments}
\begin{deflist}
	\item[username]String containing username. Must be a valid email address.
	\item[password]String containing password.
\end{deflist}

\textbf{Return Value}

True if registration is successful. False otherwise.


\subsection{Database}
Executes reads and writes to the database. Located in \url{database\_utils.py}.

\subsubsection{setup}
\textbf{Description}

Class method to create a MySQL instance at the beginning of app execution. Uses a default configuration from a DatabaseConfig object and allows custom configuration parameters. Once configured, create one connection and add it to the connection pool.

\textbf{Arguments}
\begin{deflist}
	\item[app]Flask app instance.
	\item[**kwargs]Custom configuration parameters.
	\begin{deflist}
		\item[socket]Tuple containing hostname and port of the MySQL database. Defaults are 
					 "MYSQL\_DATABASE\_HOST" environment variable and port 3306.
		\item[credentials]Tuple containing user and password to access database.
		\item[database]Name of database to use. Default is "cravr".
		\item[charset]Character set used by database. Default is "utf8".
	\end{deflist}
\end{deflist}

\textbf{Return Value}

None.

\subsubsection{execute\_query}
\textbf{Description}

Gets a connection from the connection pool, executes a query on the database, and gives the connection back. Fails if database connection cannot be established.

\textbf{Arguments}
\begin{deflist}
	\item[query]String containing MySQL query to be executed.
\end{deflist}

\textbf{Return value}

String containing the first result of the query. -1 if database connection failed.

\subsection{Recommender}
Suggests restaurants based on user and restaurant data. Located in \url{recommender.py}

\subsubsection{get\_restaurant}
\textbf{Description}

Suggests a restaurant depending on search criteria and the user's model.

\textbf{Arguments}
\begin{deflist}
	\item[user]String containing the username.
	\item[search\_params]Dictionary of search parameters to be passed to a Yelp API business search.
\end{deflist}

\textbf{Return value}

Dictionary containing the following key-value pairs:
\begin{deflist}
	\item[id]ID of the restaurant provided by Yelp API.
	\item[Name]Name of the restaurant.
	\item[Location]Tuple containing the restaurant's location coordinates.
	\item[Distance]Distance from the user's location to the restaurant's location.
	\item[Price]String containing between 1 and 4 \$ symbols.
	\item[Rating]Numerical rating of the restaurant from Yelp reviews.
	\item[Image]Link to an image of the restaurant.
\end{deflist}

\subsubsection{cache\_restaurant}
\textbf{Description}

Add recently rated restaurants to a user's cache. This keeps the user from seeing the same recommendation multiple times in a row.

\textbf{Arguments}
\begin{deflist}
	\item[user]String containing the username.
	\item[restaurant\_id]ID of the restaurant provided by Yelp API.
\end{deflist}

\textbf{Return value}

None.

\subsection{Routing}
Interacts directly with the frontend for authentication, restaurant recommendations, and user feedback. Provides restaurant/review data and routing information. Located in \url{app.py}.

\subsubsection{login}
\textbf{Description}

Uses authentication and database functionality to check if user's credentials are valid. Fails if username not found in database, hashed password does not match, or database connection cannot be established.

\textbf{Return value}

\{"result": "/"\} if authentication successful. \{"result": "/Login"\} otherwise.

\subsubsection{register}
\textbf{Description}

Uses authentication and database functionality to register a new user. Fails if username already exists in database or database connection cannot be established.

\textbf{Return value}

\{"result": "/Login"\} if registration successful. \{"result": "/Register"\} otherwise.

\subsubsection{get\_restaurant}
\textbf{Description}

Parses a user's search parameters and gives a restaurant recommendation. 

\textbf{Return value}

\{"result": RESTAURANT\_OBJECT\} where RESTAURANT\_OBJECT is the return value of the Recommender class's get\_restaurant method.

\subsubsection{rate\_suggestion}
\textbf{Description}

Take a restaurant ID along with a user's rating of "Yummy", "Maybe later", or "Yuck" to update their review list and train the recommender.

\textbf{Return value}

None.

\subsubsection{get\_reviews}
\textbf{Description}

Fetch a list of restaurants that the user needs to review.

\textbf{Return value}

\{"result": REVIEW\_LIST\} where REVIEW\_LIST is the return value of the User class's get\_reviews method.

\subsubsection{submit\_review}
\textbf{Description}

Remove a restaurant from a user's review list and use review data to train their recommender.

\textbf{Return value}

None.

\subsection{User Utilities}
Manage users' restaurant caches, review lists, and recommender training data. Located in \url{user.py} and \url{user\_data\_utils.py}

\subsubsection{User}
\textbf{Description}

Class that maintains permanent user data such as the restaurants they need to review and their recommendation model.

\textbf{Fields}
\begin{deflist}
	\item[str name] The user's username or email address.
	\item[bool is\_dirty] Boolean for whether or not the User object has been modified since it was last written to the database.
	\item[list reviews] List of Yelp business IDs (strings) that the user has clicked "Yummy" on and must review.
	\item[Model model] Model object that serves as the user's individual recommendation model.
\end{deflist}

\textbf{Methods}
\begin{deflist}
	\item[add\_review]
		\textbf{Description}\\Append a Yelp ID to the User's reviews list.\\
		\textbf{Return value}\\None.
	\item[get\_reviews]
		\textbf{Description}\\Get list of restaurants that need to be reviewed by the User.\\
		\textbf{Return value}\\List of dictionaries for each ID in the reviews list containing the following: \begin{deflist}
			\item[restaurant] Yelp object returned by looking up the ID's business details.
			\item[review] Dictionary containing the default review settings that are changed in the frontend.
		\end{deflist}
	\item[submit\_review]
		\textbf{Description}\\The User reviewed a restaurant in the frontend and it must be removed from the User's reviews list passed to the model for training\\
		\textbf{Return value}\\None.
	\item[handle\_review]
	\textbf{Description}\\Train the User's model based on the review for this restaurant.\\
	\textbf{Return value}\\None.
	\item[disliked]
	\textbf{Description}\\Adjust model weights to discourage similar restaurants from being suggested when the "Yuck" button is clicked.\\
	\textbf{Return value}\\None.
\end{deflist}

\subsubsection{UserList}
\textbf{Description}

Class that caches a dictionary in memory of unique usernames mapped to User objects.
This is a subset of all of the user data stored in the database.
When the UserList is deleted (e.g. when the app goes down), it writes all modified user data to the database in its destructor.
Its default methods have been overridden so that it behaves similar to a standard dictionary in Python.

\textbf{Fields}
\begin{deflist}
	\item[dictionary users] Mapping of unique usernames to their User objects.
	\item[bool is\_prod] Boolean that when false, disables the destructor in order to pass unit tests.
\end{deflist}

\subsubsection{read\_user\_data}
\textbf{Description}

Read the user's data from the database

\textbf{Arguments}
\begin{deflist}
	\item[str username] String for the User's username
\end{deflist}

\textbf{Return value}

A dictionary containing the user's reviews list and model if the user exists in the database, otherwise None.

\subsubsection{write\_user\_data}
\textbf{Description}

Modify the user's data in the database

\textbf{Arguments}
\begin{deflist}
	\item[str username] String for the User's username
	\item[User data] User object containing data that will be replace the value at the key 'username' in the database if it exists, otherwise a new entry in the database will be created.
\end{deflist}

\textbf{Return value}

True if write successful, False otherwise

\subsection{Yelp Fusion API}
Gets restaurant data from Yelp Fusion API. Located in \url{yelp\_api\_utils.py}

\subsubsection{business\_search}
\textbf{Description}

Find restaurants matching a provided search term, location, and other custom parameters.

\textbf{Arguments}
\begin{deflist}
	\item[term]String containing the search term.
	\item[location]Location of the user. Can be provided as a string (e.g. "New York City") or as a tuple of coordinates (e.g. (40.4167, -86.8753)).
	\item[**kwargs]Custom search parameters.
	\begin{deflist}
		\item[radius]Search radius from provided location in miles. If this value converts to over 40km (maximum for Yelp API), the radius will be set to 40km.
		\item[price]String containing between 1 and 4 \$ symbols denoting the desired price range.
		\item[open\_now]Boolean value for whether or not the restaurant is currently open. Default is True.
	\end{deflist}
\end{deflist}

\textbf{Return value}

\subsubsection{business\_details}
\textbf{Description}

Get details about a restaurant from Yelp Fusion API.

\textbf{Arguments}
\begin{deflist}
	\item[business\_id]ID of the business provided by Yelp API.
\end{deflist}

\textbf{Return value}

Dictionary containing details for the given business ID.

\subsubsection{business\_reviews}
\textbf{Description}

Get reviews for a restaurant from Yelp Fusion API.

\textbf{Arguments}
\begin{deflist}
	\item[business\_id]ID of the business provided by Yelp API.
\end{deflist}

\textbf{Return value}

Dictionary containing reviews for the given business ID.

\end{document}